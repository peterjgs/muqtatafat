% Options for packages loaded elsewhere
\PassOptionsToPackage{unicode}{hyperref}
\PassOptionsToPackage{hyphens}{url}
%
\documentclass[
]{book}
\usepackage{amsmath,amssymb}
\usepackage{iftex}
\ifPDFTeX
  \usepackage[T1]{fontenc}
  \usepackage[utf8]{inputenc}
  \usepackage{textcomp} % provide euro and other symbols
\else % if luatex or xetex
  \usepackage{unicode-math} % this also loads fontspec
  \defaultfontfeatures{Scale=MatchLowercase}
  \defaultfontfeatures[\rmfamily]{Ligatures=TeX,Scale=1}
\fi
\usepackage{lmodern}
\ifPDFTeX\else
  % xetex/luatex font selection
\fi
% Use upquote if available, for straight quotes in verbatim environments
\IfFileExists{upquote.sty}{\usepackage{upquote}}{}
\IfFileExists{microtype.sty}{% use microtype if available
  \usepackage[]{microtype}
  \UseMicrotypeSet[protrusion]{basicmath} % disable protrusion for tt fonts
}{}
\makeatletter
\@ifundefined{KOMAClassName}{% if non-KOMA class
  \IfFileExists{parskip.sty}{%
    \usepackage{parskip}
  }{% else
    \setlength{\parindent}{0pt}
    \setlength{\parskip}{6pt plus 2pt minus 1pt}}
}{% if KOMA class
  \KOMAoptions{parskip=half}}
\makeatother
\usepackage{xcolor}
\usepackage{graphicx}
\makeatletter
\def\maxwidth{\ifdim\Gin@nat@width>\linewidth\linewidth\else\Gin@nat@width\fi}
\def\maxheight{\ifdim\Gin@nat@height>\textheight\textheight\else\Gin@nat@height\fi}
\makeatother
% Scale images if necessary, so that they will not overflow the page
% margins by default, and it is still possible to overwrite the defaults
% using explicit options in \includegraphics[width, height, ...]{}
\setkeys{Gin}{width=\maxwidth,height=\maxheight,keepaspectratio}
% Set default figure placement to htbp
\makeatletter
\def\fps@figure{htbp}
\makeatother
\setlength{\emergencystretch}{3em} % prevent overfull lines
\providecommand{\tightlist}{%
  \setlength{\itemsep}{0pt}\setlength{\parskip}{0pt}}
\setcounter{secnumdepth}{-\maxdimen} % remove section numbering
\usepackage{arabluatex}
\usepackage{fontspec}
\ifLuaTeX
  \usepackage{selnolig}  % disable illegal ligatures
\fi
\usepackage{bookmark}
\IfFileExists{xurl.sty}{\usepackage{xurl}}{} % add URL line breaks if available
\urlstyle{same}
\hypersetup{
  pdftitle={A personal web of information},
  hidelinks,
  pdfcreator={LaTeX via pandoc}}

\title{A personal web of information}
\author{}
\date{\vspace{-2.5em}2024-12-09}

\begin{document}
\frontmatter
\maketitle

\mainmatter
\begin{verbatim}
## Loading required package: lualatex
\end{verbatim}

\begin{verbatim}
## Warning in library(package, lib.loc = lib.loc, character.only = TRUE,
## logical.return = TRUE, : there is no package called 'lualatex'
\end{verbatim}

\chapter{A personal web of
information}\label{a-personal-web-of-information}

مُقْتَطَفات مَعلومات

\emph{A selection of information}

\section{Arab caliphs and the Shi'a
split}\label{arab-caliphs-and-the-shia-split}

From: Hitti, \emph{History of the Arabs}

The first four caliphs were known as the ``orthodox caliphs''. They held
sway from Medina.

\begin{itemize}
\item
  Abu Bakr (r. 632-634)
\item
  'Umar (r. 634-644)

  \begin{itemize}
  \tightlist
  \item
    appointed a six-member board of electors to decide on next caliph;
    stated that his son was not to be his successor
  \item
    this shoed that the old Arabian idea of the tribal chief had
    triumphed over the idea of the hereditary ruler
  \end{itemize}
\item
  'Uthman (r. 644-656)

  \begin{itemize}
  \tightlist
  \item
    there was a protest against 'Uthman and he was killed
  \item
    'Aishah, Muhammad's favourite wife, had been involved in it in some
    way
  \end{itemize}
\item
  'Ali (r. 656-661)

  \begin{itemize}
  \tightlist
  \item
    'Ali was Muhammad's first cousin and husband of Muhammad's favourite
    daughter, Fatimah; they had two sons, Hasan and Husayn
  \item
    the party he represented had long maintained 'Ali was the legitimate
    successor to Muhammad!
  \item
    almost the whole Muslim world accepted his succession as caliph
  \item
    some were opposed to him - 'Aishah, now ``the mother of the
    believers'', joined up with these opponents in Basra; she hated him
    because of an earlier personal dispute between them where he'd
    questioned her integrity
  \item
    the two sides confronted each other at Basra; the ``battle of the
    camel'' (named from 'Aisha's mount) ensued; 'Ali won; 'Aisha, still
    treated as ``first lady'', was sent back to Medina
  \item
    'Ali moved his capital to Kufa not long after he became caliph
  \end{itemize}
\end{itemize}

This was the first ``civil war''.

Mu'awiyah, governor of Syria and relative of 'Uthman, emerged to avenge
the killing of 'Uthman.

Mu'awiyah refused to accept 'Ali as caliph and challenged 'Ali to
produce 'Uthman's killer, saying that, if he didn't produce the killer,
he'd be an accomplice to the murder and ineligible to be caliph.

This also raised the question about whether Iraq or Syria would dominate
Islamic affairs.

The two sides faced off near Raqqa (in present day northern Syria) in
657. There was a very drawn-out arbitration process, the details of
which are unclear. In this process, which ended in 659, 'Ali lost status
and Mu'wiyah gained some.

After 'Ali was killed in 661, Mu'awiyah became caliph, having been the
caliph in waiting (but already accepted by some) since 659. He'd
actually been proclaimed caliph in Jerusalem in 660. 'Ali was buried at
al-Najaf, now a major Muslim pilgrimmage centre.

Mu'awiyah (r. 661-680) established Damascus as the capital of the Muslim
empire. He nominated his son, Yazid, as his successor. This introduced
the hereditary principle into the caliphate.

Mu'awiyah was the first of the Umayyad caliphs (661-750).

The Iraqis refused to accept Mu'awiyah as caliph and put up Hasan, 'Ali
and Fatimah's elder son, as 'Ali's true successor. Hasan wasn't
particularly interested and was bought off by Mu'awiyah.

When Yazid succeeded his father, Mu'awiyah in 680, Iraqis pushed Husayn
to challenge Yazid.

On 10th Muharram, 680, Husayn and a group of a couple of hundred
relatives and followers were confronted by forces loyal to Yazid.
Husayn's group refused to surrender and were all killed.

Shi'ism was born on 10th Muharram, as a result of Husayn's death.

Shi'ites hold strongly to the idea that the imamship began with 'Ali.

\backmatter
\end{document}
